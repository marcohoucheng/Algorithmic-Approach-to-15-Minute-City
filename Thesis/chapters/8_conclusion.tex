\chapter{Conclusion} \label{conclusion}

This thesis has presented a novel algorithm to measure the accessibility of amenities in urban areas. The algorithm is based on the concept of the ``15-Minute City'', which is a city where all amenities are reachable within 15 minutes of walking or cycling. The algorithm is computationally efficient, as the running time on a single core is less than 10 seconds even for a city as large as London. The algorithm is flexible and can be used to measure the accessibility of different types of amenities, such as coffee shops, post offices, and supermarkets. The algorithm can be used in many approaches to the ``15-Minute City'' concept, as many existing solutions rely on calculating an area reachable within 15 minutes (or $t$ minutes) from a given location as the first step.

% Existing papers do promote reproducitbility