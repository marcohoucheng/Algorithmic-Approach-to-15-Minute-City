\chapter{Problem Statement} \label{problem_statement}

In the previous section \ref{review}, several studies have discussed the needs for standardisation in the methodology used in 15-Minute City. In particular, Pezzica et al. argued that \say{the absence of standardised modelling protocols imposes significant limitations on the application of Minute City models, hinders synchronic comparisons, and can indirectly foster the formulation of exclusive policies and inconsistent planning decisions} \cite{Pezzica_Altafini_Mara_Chioni_2024}. Lima and Costa also noted that \say{introducing computational approaches to 15-Minute City design presents significant challenges and potential bottlenecks. On the other hand, exploring these challenges as opportunities for inserting new research is also possible since the theme is rising}, some of these challenges are ``data availability and quality", ``computational cost'' and ``adaptability'' \cite{lima_quest_2023}. Furthermore, Marchigiani et al. found that the approach to 15-Minute City needs to be changed and adapted to each location case by case \cite{marchigiani_urban_2022}.

Furthermore, a number of other literature works discussed in Chapter \ref{review} have mentioned the use of a graph search algorithm to find the 15-Minute City or the travelling time from the source location of interest. Notably, Caselli et al. (Section \ref{caselli_exploring_2022}, \cite{caselli_exploring_2022}) defined 15-Minute City from the ``neighbour cores'' and Rhoads et al. (Section \ref{rhoads_inclusive_2023}, \cite{rhoads_inclusive_2023}) used a modified Dijkstra's algorithm to compute a 15-Minute City on side-walk networks with a walkability analysis. The grid tessellation approaches used by Gaglione et al. (Section \ref{gaglione_urban_2022}, \cite{gaglione_urban_2022}) and Olivari et al. (Section \ref{olivari_are_2023}, \cite{olivari_are_2023}) also searched for travelling time in a spatial space.

With these in mind, the aim of this thesis is to develop a general, adaptable algorithm to identify the 15-Minute City, where a person can travel to all their essential needs within 15 minutes from their home. In Computer Science, Graph Theory is a well-established field and there exist many efficient graph search algorithms. In this thesis, we will develop an algorithm by adapting various techniques from existing algorithms, along with inspirations from a number of approaches explored in literature which we discussed in the previous section \ref{review}.

The solution proposed in this thesis should be able to support different types of graphs and service locations. The designed algorithm will focus on the `computational cost'' and ``adaptability'' problems listed by Lima and Costa \cite{lima_quest_2023}. The algorithm should allow for an arbitrary set of service types. The edge weights in the graph use time unit, as this promotes the freedom for the users to incorporate different characterise to the roads which could affect the travelling time, such as the the street width, slope, and/or the mode of transportation. Therefore, the solution to the problem stated in this section can be considered as an adoption or a modification to the existing approaches to the 15-Minute City problem, specifically with an improvement of the computational efficiency in mind.

Formally, the algorithm to the 15-Minute City problem should satisfy the following properties.

% n, m graph size
% s source
% u, v, w notation of vertices
% t time
% l vertex of services in a node
% r reachable within t minutes
% i i^th service type
% p number of service types
% q_i number of locations of service type i

\rule{\textwidth}{0.4pt}

\textbf{Inputs}
\begin{enumerate}
    \item A graph $G(V,E)$ representing the area of which $t$-Minute City is computed. $V$ is the set of vertices representing locations within the area, and $E$ is the set of weighted, undirected edges such that $E\subseteq V\times V$ and the weights $w:E\rightarrow\mathbb{R}_{+}$ of the edges are proportional to the time required to travel along the corresponding edge, in minutes.
    \item For every vertex $v\in V$, $v$ contains the label $v.l\in\{0,1\}^p$, where $v.l[i]$ represents the availability of service type $i$ of the location.
    \item A time threshold $t$ in minutes.
\end{enumerate}

\textbf{Output}

A set of vertices $R\subseteq V$ which can reach to at least one location of each service type within $t$ minutes. i.e. denote $d(v,w)$ as the shortest path distance between $v$ and $w$, we have $$\forall v\in R:\forall i\in[1,p],\enspace\exists w\in V,\enspace w.l[i]=1:d(v,w)\leq t$$ Moreover, denote $v.r[i]\in\{0,1\}^p$ as a binary vector indicating whether $v$ can reach a location of service type $i$ within $t$ minutes, such that $$v.r[i]=1\iff\exists w\in V,\enspace w.l[i]=1:d(v,w)\leq t$$ then, we have $$\forall v\in R,\enspace v.r = \mathbf{1}$$

\rule{\textwidth}{0.4pt}

In this setting, a location $v\in V$ in this graph could be a location of an amenity of interest, or a road junction, such that $v.l\in\{0\}^p$. More specifically, given a list of services of $p$ distinct types and their locations, each location can be inserted into the graph by

\begin{itemize}
    \item Labelling the closest node in the graph by the service type, or
    \item Adding a vertex to $V$ along an existing edge in $E$, in such a way that the distance from the vertex to the nearest road junction is minimised.
\end{itemize}