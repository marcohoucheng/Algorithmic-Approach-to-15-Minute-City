\chapter{Introduction} \label{intro}

The 15-Minute City is an urban design concept that promotes accessibility to essential urban functions within a 15-minute travel from the homes of residents. The concept was first proposed by Moreno in 2016 as a solution to build safer, more resilient, sustainable and inclusive cities and to harmonise the notion of Smart Cities \cite{moreno_introducing_2021}. In 2021, Moreno et al. discussed the relationship between the concept of the “15-Minute City” to Urban Planning Pandemic Response, its Emerging Variations (i.e. 20 minute city \cite{capasso_da_silva_accessibility_2019}), walkable neighbour \cite{weng_15-minute_2019} smart cities.

The formal “15-Minute City Concept” proposed by Moreno et al. argues that residents should be able to enjoy a higher quality of life where they will be able to effectively fulfil six essential urban social functions to sustain a decent urban life. These include

\begin{multicols}{2}
    \begin{enumerate}
        \item Living
        \item Working
        \item Commerce
        \item Healthcare
        \item Education
        \item Entertainment
    \end{enumerate}
\end{multicols}

The authors then proposed the “modified 15-Minute City” framework, depicting the four identified dimensions that could be incorporated with the already existing one proposed.

\begin{multicols}{2}
    \begin{enumerate}
        \item Density
        \item Proximity
        \item Diversity
        \item Digitisation
    \end{enumerate}
\end{multicols}
Since COVID-19 pandemic, this topic has gained a tremendous amount of attention and has growth exponentially in popularity \cite{lima_quest_2023,allam_theoretical_2022} among various research fields in literature, with urban design being the most studied \cite{lima_quest_2023}. The concept of the 15-Minute City has been studied and shown that it brings benefits to the society including environmental, social, and economical impacts \cite{allam_theoretical_2022}. Although the 15 minutes threshold has attracted the most attention, the notion of the $t$-Minute City has also been considered \cite{moreno_introducing_2021}. Some examples are the 20-Minute City in Tempe, Arizona \cite{capasso_da_silva_accessibility_2019}, the 30-Minute City in Sydney, Australia \cite{sarkar_measuring_2020} and Olivari et al. studied 5 to 60 Minute City in Ferrara, Italy \cite{olivari_are_2023}. 

Most 15-Minute City studies in literature have employed data-driven approaches, these studies focus on a specific city or location and their methodologies are therefore only applicable to the specific location of interest. For example, Weng et al. \cite{weng_15-minute_2019} and Olivari et al.\cite{olivari_are_2023} relied on census data on population density for their respective studies which may not be available for some countries and regions. The choice of amenities to be included in the 15-Minute City is subjective and varies from one study to another. There is also a lack of research on the complexities and computational challenges of implementing the 15-Minute City concept in practice.

In 2023, Lima and Costa noted that computational approaches to 15-Minute City design presents significant challenges such as “data availability and quality”, “computational cost” and “adaptability” \cite{lima_quest_2023}. In this thesis, we aim to address the latter two challenges mentioned by Lima et al. \cite{lima_quest_2023}. We propose an algorithmic approach to the 15-Minute City concept that is general, adaptable and efficient. We will study a number of existing relevant graph search algorithms and the complexities of the proposed algorithm. We will implement the proposed algorithm on Rust programming language and evaluate its performance on a number of real-world locations. We will also compare the "15-Minute City" generated by the proposed algorithm with existing solutions in literature.