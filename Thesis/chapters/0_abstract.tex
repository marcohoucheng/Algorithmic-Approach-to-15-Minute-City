% Insert abstract
\begin{Abstract}
\begin{changemargin}{1cm}{1cm}

The 15-Minute City is an urban planning concept introduced in the last decade that promotes accessibility. It emphasizes that residents should be able to meet basic needs—such as groceries, education, healthcare, and leisure—within a 15-minute travel time from their homes. The concept aims to deliver environmental, social, and economic benefits by reducing reliance on automobiles, encouraging active transportation, and enhancing residents’ quality of life through easy access to essential services and amenities. The concept has gained traction during the COVID-19 pandemic, which highlighted the significance of local services and amenities in urban settings.

\vspace{0.5cm}

The 15-Minute City concept has been explored across various research fields, including urban planning, transportation, and environmental science. Within the field of Computer Science, although methodologies have been developed for the topic, a generalised purpose algorithmic approach to identify a 15-Minute City is still lacking. Most existing studies are data-driven, focusing on specific cities with solutions that are often neither algorithmic nor generalised.

\vspace{0.5cm}

This thesis aims to develop a general, adaptive, and efficient algorithm to identify city areas that can be classified as a 15-Minute City. It examines several existing algorithms for graph data structures, such as Breadth-First Search, Dijkstra’s algorithm, Johnson’s algorithm, and their variations. The proposed ”15-Minute City algorithm” synthesises ideas and techniques from these algorithms to offer a comprehensive and efficient solution for determining 15-Minute City areas.

\end{changemargin}
\end{Abstract}