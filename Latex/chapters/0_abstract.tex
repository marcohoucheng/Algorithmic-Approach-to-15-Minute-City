% Insert abstract
\begin{Abstract}
\begin{changemargin}{1cm}{1cm}

The 15-Minute City is an urban planning concept introduced in the last decade, it promotes accessibility of basic needs within a short distance in urban areas and aims to provide environmental, social, and economical benefits to society. The concept has gained popularity during the COVID-19 pandemic, as it has highlighted the importance of local services and amenities in urban areas. The 15-Minute City is based on the idea that residents should be able to access basic needs, such as groceries, education, healthcare, and leisure, within a 15-minute travelling time from their homes.

\vspace{0.5cm}

The concept of the 15-Minute City has been studied in different research fields, including urban planning, transportation, and environmental science. However, there is not a general methodology to determine the areas of a city that can be classified as a 15-Minute City. Most of the existing studies are data driven and focus on specific cities, their solutions are often not algorithmic nor generalised.

\vspace{0.5cm}

The aim of this thesis is to provide a general, adaptive and efficient algorithm to determine the areas of a city that can be classified as a 15-Minute City. A number of existing algorithms on graph data structure are studied, including Breadth-First Search, Dijkstra's algorithm, Johnson's algorithm and some of their variations. The ``15-Minute City algorithm" is then proposed, by combining various ideas and techniques from various algorithms to create a comprehensive and efficient solution that can be used to determine the areas of a city that can be classified as a 15-Minute City.

\end{changemargin}
\end{Abstract}