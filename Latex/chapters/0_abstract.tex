% Insert abstract
\begin{Abstract}
\begin{changemargin}{1cm}{1cm}

15 Minute City is a new idea in the literature world. First proposed by Moreno in 2016 as an idea to improve sustainability, economical ... benefits. However, the concept of 15 Minute City is still not well defined. In this thesis, we aim to provide a general, adaptable algorithm to identify the 15 Minute City, where a person can travel to all their needs within 15 minutes from their home. The algorithm takes a graph data structure $G(V,E)$ representing the entire area of which the 15 Minute City is computed, and a time threshold $t$ in minutes as inputs. The output is a set of vertices $R\subseteq V$ of vertices which can reach at least one location of each service type within $t$ minutes. The algorithm uses a modified graph search algorithm to find the 15 Minute City with the service locations of each type as the source nodes, rather than searching from every vertex in the graph. This approach is expected to be more efficient as the number of service locations is expected to be far smaller than the number of vertices in the graph. The solution proposed in this thesis is designed to be more general and adaptable to different types of graphs and service locations. The next section will discuss the proposed solutions.

\end{changemargin}
\end{Abstract}