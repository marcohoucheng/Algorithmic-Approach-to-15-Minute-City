In this section, we will discuss a number of previous works of 15-Minute City. As this topic have been studied in various research fields in literature, we will group the works according to their methodologies and approaches, which include graph representation, grid tessellation, flow data, Walk Score, and other works. Table \ref{tab:previous_works} provides a summary of these studies mentioned in this section excluding other works, specifically the methodologies discuessed and used in determining and finding the so-called "15-Minute City". Other works will cover studies about the general concept of the "15-Minute City". The following sections are the detailed descriptions of each study and methodology used.

\begin{table}[htbp]
    \begin{center}
        \caption{Summary of relevant previous works}
        \label{tab:previous_works}
        \begin{tabular}{ |c|c|c| }
            \hline
            \textbf{Approach} & \textbf{Work} & \textbf{Description} \\
            \hline
            \hline
            Graph Representation & \makecell{\cite{barbieri_graph_2023}, \cite{caselli_exploring_2022},\\\cite{rhoads_inclusive_2023}} & \makecell{Maps represented by graphs\\as a mathematical structure}\\
            \hline
            Grid Tessellation & \makecell{\cite{gaglione_urban_2022}, \cite{olivari_are_2023},\\\cite{Pezzica_Altafini_Mara_Chioni_2024}} & \makecell{Maps divided according to various\\shapes and 15-MC calculations are\\applied to each area independently} \\
            \hline
            Flow Data & \cite{zhang_towards_2023}, \cite{sarkar_measuring_2020} & \makecell{Use foot travel data to\\incorporate human mobility patterns} \\
            \hline
            Walk Score & \cite{walkscore}, \cite{weng_15-minute_2019} & \makecell{Proprietary methodology\\based on sets of specific factors} \\
            \hline
        \end{tabular}
    \end{center}
\end{table}

\section{Graph Representation}

\subsection{Graph Representation of the 15-Minute City: A Comparison between Rome, London, and Paris} \label{barbieri_graph_2023}

Barbieri et al. defined the general $t$-minute city (hereafter $t$-MC) on an urban graph with respect to a given set of services \cite{barbieri_graph_2023}. The urban graph is represented by a planer graph $G(V,E)$ (when a connected graph can be drawn without any edges crossing).

The nodes of $G$ are the intersections of the roads, and the lengths of the edges are proportional to the travel time with a coefficient, which is the speed of the pedestrians. Services $f\in V$ are then placed by adding an extra node to the graph, or use the nearest junction (as per Euclidean distance).

Given a list of $N_i$ services of type $i$, $C^i_k=(C^i_1,C^i_2,...,C^i_{N_i})$ are the nodes that reach $f^i$ in less than 15 min. The set of vertices that form a 15-MC with respect to services of type $i$ is given by the union of all these vertices

$$ C^i=\bigcup_{k=1}^{N_i}C^i_k\subseteq V $$

If services $f_i$, $f_j$ of the same type are far enough, it is possible that $C_i\cap C_j=\emptyset$. The authors noted that these “gaps” can be recognised as “the places where it is necessary to intervene to reconnect”. Then for $K$ types of services, the set of the 15-minute vertices is

$$C=\bigcap^K_{i=1}C^i=\bigcap^K_{i=1}\left(\bigcup^{N_i}_{j=1}C^j_k\right)\subseteq V$$

Finally, the authors define $G_C$ as the graph induced on $G$ by the vertices $C$, the 15-MC graph is the subgraph $G_C$ of the urban graph $G$. To formalise, the set $C$ depends on the travel time $t$, the service matrix $\mathbf s$, and the travel speed $v$, the graph of the 15-MC can be defined as

$$ G_C=G_C(C,E_C;t,\mathbf s,v)\subseteq G $$

A metric/ratio $\gamma(r,x_0;t,\mathbf s,v)$ was also defined as a function of the radius $r$ with respect to a given origin $x_0$, which can be used to characterise the 15-MC and compare different cities or areas of the same city, where

$$ \gamma(r,x_0;t,\mathbf s,v)\equiv|C(r,x_0;t,\mathbf s,v)|/|E| $$

A possible generalisation of the index was then suggested which takes into account the properties or weights $w(e),e\in E$ of each edge, such as the length of the path, the population density or the slope of the streets

$$ \gamma=\frac{\sum_{c\in C}w(c)}{\sum_{e\in E}w(e)} $$

If $w(\cdot)$ is the population density, the edges of the parks and archaeological area have $w = 0$, and the index $\gamma$ is not biased.

For the case study, the authors used a "shortest path search algorithm” for graph search \cite{dijkstra_note_1959}.

\textit{\textbf{Remark}: This paper transform map data into a planar graph, the algorithm starts from services rather than houses which reduces computational complexity. The authors referenced an article about Dijkstra algorithm for graph search but their implementation was not mentioned.}

\subsection{Exploring the 15-minute neighbourhoods} \label{caselli_exploring_2022}

According to the authors (Caselli et al.), \say{in the proposed study, the 15-Minute City theme is addressed with an analytical model designed and developed using GIS to assess existing conditions of accessibility to neighbourhood services for all the resident social groups.} \cite{caselli_exploring_2022}

The GIS-model is implemented by improving and integrating a Territorial Information system (managed with ArcGIS software). Extracting the pedestrian paths feature class to generate a link-node graph with all walking routes available. The model also considers that users “might choose to walk along road margins or cross in the proximity of road intersections.”

The paper studied the area covered that can be travelled to \say{neighbour cores} within 15 minutes by the following calculation:

$$\text{Length}(km) / (3 km/h \times 60 min) + \text{DF} (min)$$

where $\text{DF}$ is the delay factor at crossings (20 and 30 seconds for non-signalised and 40 and 60 seconds for signalised).

The authors then compare this area with its population distribution to study the proportion of population covered by the 15-MC.

\textit{\textbf{Remark:} The authors did not define the neighbour cores, only by saying \say{urban nodes well served by necessities shops and services, such as supermarkets, grocery stores, bars, drugstores, and banks.} However, using such nodes to calculate for 15-Minute City contributes to faster running time. The actual 15-Minute City search approach was not mentioned.}

\subsection{The inclusive 15-Minute City: Walkability analysis with sidewalk networks} \label{rhoads_inclusive_2023}

The paper proposed a framework for assessing multi-factor walkability on a sidewalk network model \cite{rhoads_inclusive_2023}. The sidewalk network model is defined as a graph where the nodes are intersections or crosswalks, the edges have 3 types: sidewalks, crosswalks, and pedestrian-only paths and 4 attributes: length, width, slope, and pedestrian hazard. The pedestrian-only paths includes pedestrianised streets, living streets, and paths through parks and plazas. Pedestrian hazard is a metric to describe how dangerous each sidewalk segment by using a fine-grained map of estimated pedestrian safety in Barcelona \cite{bustos_explainable_2021} and by exploiting Deep Learning tools. The resulting network is denser than the road network (approximately 4 to 1 in both nodes and edges). Each node of the graph has also been assigned a population according to census data.

This network is then simplified by a percolation analysis according to the sidewalks' properties (i.e., width, slope, or hazard). The authors then noted that the analysis on the 1260m that can be travelled in 15 min at a walking speed of 1.4 m/s, in accordance with literature \cite{bosina_estimating_2017}. Using government data, the authors selected a list of services.

To find such a set of links, we extended the classic Dijkstra algorithm to

\begin{enumerate}
\item explore all nodes within the threshold time from a single source, and
\item to record all edges that can be traversed within the threshold, not only the ones that form part of a shortest path.
\end{enumerate}

A formal description can be found in Section S3 of the Supplementary Materials, while the implementation was done in Python, using the {igraph} library \cite{igraph}.

The authors then studied the impact of population size of the largest and second largest connected component of Barcelona's sidewalk networks by changing the parameters of percolation analysis.

\textit{\textbf{Remark:} This model takes hazard as a factor. The percolation analysis is a main focus of the study. The Supplementary notes also included the modification of Dijkstra algorithm.}

\section{Grid Tessellation}

\subsection{Urban accessibility in a 15-Minute City a measure in the city of Naples, Italy} \label{gaglione_urban_2022}

The authors (Gaglione et al.) in this paper proposed a 4 steps methodology through a GIS environment to define the areas accessible in 15 minutes \cite{gaglione_urban_2022}.

\begin{enumerate}
\item With a systemic approach, 17 variables have been identified by
\begin{itemize}
    \item The characteristics of the population.
    \item The characteristics of urban fabrics, in particular their shape.
    \item The physical characteristics relating to safety, amenities and pleasantness of the pedestrian.
    network.
\end{itemize}
\item The relationships among different groups of characteristics were identified through a (Pearson) correlation analysis to remove some variables.
\item Relating the demand (users) to the supply (local urban services). The authors used a proximity analysis by calculating the Euclidean distances from the centroids of the census sections to the related closer local urban services, then study how users can move along the pedestrian network by labelling 13 characteristics on each pedestrian path. \say{On the basis of the travel times defined on each link of the pedestrian network and the distribution and location of all the local services examined, urban areas accessible in 15 minutes have been defined.}
\item The population density is then compared with the 15-minute accessible areas. Individual age groups are also studied in this context.
\end{enumerate}

The authors then explored the effect of choosing different grid size, shapes etc. in terms of the results of the minute city.

\textit{\textbf{Remark:} The authors did not specify walking speeds or any algorithms used in calculating travel times.}

\subsection{Are Italian cities already 15-minute?} \label{olivari_are_2023}

The authors (Olivari et al.) of this paper proposed a data-driven approach solution by defining the NExt proXimity Index (NEXI), which “exploits the data to answer the question: \say{Which parts of your city or town already follow the 15-minute model?} \cite{olivari_are_2023}

With a selected list of service categories, and the nodes of the road network (the intersection points of the network geometries) and the points of interest (the geographical location of the various services). For each node the algorithm computes the time needed to reach – at an average walking speed - the closest point of interest of any given category, being constrained to move only on roads accessible to pedestrians. More in details, the time needed to reach the points of interest is computed as $t = l / s$ , where:

\begin{itemize}
\item $l$ is the length of the shortest route to the PoI (Point of Interest), on a road network made only of walkable roads,
\item $s$ is the approximate walking speed of an average person, that is 5 km/h.
\end{itemize}

If all categories can be reached within 15 minutes, the node is then considered to be a 15-minute node. Using a 5 km/h speed, the maximum reachable distance in 15 minutes is 1250m. \say{if the average time to reach all the categories from the nodes in that area is lower or equal to 15 minutes}.

The algorithm computes the level of proximity of a given area as the mean of the levels of proximity of the nodes inside that area. Therefore, an area is 15-minute if the average time to reach all the categories from the nodes in that area is lower or equal to 15 minutes. The authors used hexagons with a diameter of 250 meters as the smallest resolution unit.

3 indices are proposed by the authors:

\begin{enumerate}
\item The NEXI-Minutes assigns to each category for each area a value of time which is the average time to reach each category.
\item The NEXI-Global takes inspiration from the Walk Score methodology, measuring the global proximity to all service categories on a scale that goes from $0 - 100$, where 0 means that none of the categories is at least within a 30-minute walk, while 100 means that all categories are within a 15-minute walk and all values in between describe an intermediate situation.
\item A discomfort index which takes population into account, where
$$\text{Discomfort} = (100-\text{Global})\times\text{Population}$$
\end{enumerate}

\textit{\textbf{Remark:} This paper takes the population distribution into account, only considering walking (but can be changed easily by varying the speed parameter). It does not use graph representation and there are no mentions of graph search algorithm. Though the authors mentioned that the algorithm could be sped up by using  "parallelisable algorithm”.}

\subsection{Travel-time in a grid: modelling movement dynamics in the ``minute city"}

The authors claimed to “provide initial insights and recommendations for developing more robust 15-Minute City models” and emphasised “the importance of technical modelling steps in determining the mapping outputs which support the assessments of 15-minute cities” \cite{Pezzica_Altafini_Mara_Chioni_2024}. In the paper, they experimented on evaluating grid-based methods and identified four noteworthy variables: 

\begin{enumerate}
    \item Grid tessellation choices
    \item Software application pick
    \item Speed selection for travel-time calculations
    \item Classification rules' adoption for mapping urban functions against mapped amenities
\end{enumerate}

\say{The absence of standardised modelling protocols imposes significant limitations on the application of MC models, hinders synchronic comparisons, and can indirectly foster the formulation of exclusive policies and inconsistent planning decisions. Hence, it is crucial to undertake concerted efforts towards standardisation, including by bridging the gap between the planning practice and software development communities, to effectively address the existing challenges in MC modelling and representation. This entails establishing shared modelling protocols, algorithms (across various software applications), and standardised data inputs, all of which can substantially enhance the consistency and reliability of grid-based MC assessments focused on travel-time estimates. Ultimately, advancing research, fostering collaboration, and promoting knowledge sharing, can elevate the quality of evidence generated through spatial analysis, leading to better-informed decisions in MC planning.}

\textit{\textbf{Remark:} The authors concluded the article with the paragraph above, which highlights the importance of having a standardised method in calculating 15-Minute City. The results of the thesis can be used to be a harmonised solution to this.}

\section{Flow Data}

\subsection{Towards a 15-minute city: A network-based evaluation framework.} 

In the context of 15-Minute City, it is suggested that the allocation of facilities should account for how people access and use local service and amenities rather than merely considering population size \cite{chai_new_2020}.

The authors in this paper proposed a methodological framework for evaluating 15-Minute City based on network science approaches. The paper proposes a network-based approach to evaluating 15-Minute City \cite{zhang_towards_2023}. This approach differs from mostly used accessibility measurements by accounting for human mobility patterns.

The network-based evaluation framework contains 3 parts:

\begin{enumerate}
    \item Optimal mobility network is estimated based on the spatial distribution of urban amenities and population using a maximum flow algorithm.
    \item The actual origin-destination network is obtained using mobile phone signalling data.
    \item The differences between actual origin-destination network and the optimal network are measured to provide insights on the extent to which human mobility patterns, as a reflection on the usage patterns of urban amenities, match or do not match the schemes of urban planning and construction.
\end{enumerate}

The authors then applied the framework model to a case study in Nanjing, China.

\subsection{Measuring polycentricity via network flows, spatial interaction and percolation}

This paper studied polycentricity based on inflow and outflow trip data and considered 3 network-based centricity metrics \cite{sarkar_measuring_2020}. In particular, accessibility-based centricity index computes the total number of jobs available and the number of works available within a time threshold from a location. The time threshold was set to 30 in the study to align with the polycentricity-inspired masterplan proposed by The Greater Sydney Commission (GSC) in 2018 which applies to the Sydney-GMR (Sydney Greater Metropolitan Region), noting that \say{access to jobs, goods and services is provided to the community in three largely self-contained regions.}

\section{Walk Score}

\subsection{Walk Score}

Walk Score is a measure of how walkable an address is based on the distance and availability of nearby amenities, such as grocery stores, restaurants, schools, parks, etc. The higher the Walk Score, the more walkable the location is.

According to the Walk Score methodology \cite{walkscore}, for each address, hundreds of walking routes to nearby amenities are analysed. Points are awarded based on the distance to amenities in each category. Amenities within a 5 minute walk (0.25 miles) are given maximum points. A decay function is used to give points to more distant amenities, with no points given after a 30 minute walk. Walk Score also measures pedestrian friendliness by analysing population density and road metrics such as block length and intersection density. Data sources include Google, Factual, Great Schools, Open Street Map, the U.S. Census, Localise, and places added by the Walk Score user community.

The Walk Score calculation can be summarised into 4 steps

\begin{enumerate}
    \item Assigning raw weights for selected amenities
    \item Calculating distances from each location (community, data from government) to the selected amenities
    \item Computing the total scores based on the distances and modifying the scores according to decay factors (e.g. street intersections and block length)
    \item Normalising scores to 0–100
\end{enumerate}

Walk Score ranges from 0 to 100, with the following descriptions:

\begin{itemize}
    \item 90–100: Walker's Paradise. Daily errands do not require a car.
    \item 70–89: Very Walkable. Most errands can be accomplished on foot.
    \item 50–69: Somewhat Walkable. Some errands can be accomplished on foot.
    \item 25–49: Car-Dependent. Most errands require a car.
    \item 0–24: Car-Dependent. Almost all errands require a car.
\end{itemize}

\textit{\textbf{Remark:} The Walk Score algorithm is not open source. However, it considers walking distance between 5 to 30 minutes. Its calculation has been validated \cite{carr_validation_2011}}

\subsection{The 15-Minute Walkable Neighbourhoods: Measurement, Social Inequalities and Implications for Building Healthy Communities in Urban China}

In this paper, the authors (Weng et al.) noted some of the limitations of the Walk Score calculation \cite{weng_15-minute_2019}, such as

\begin{enumerate}
    \item It targets at overall population and the walking demands of different pedestrian groups have not been included in the assessment.
    \item The decay effect of amenity varies greatly among population groups and categories of amenities.
    \item Actual traffic situation has not been considered when calculating distances based on Euclidean distance.
\end{enumerate}

They proposed a modified method to measure 15-min walkable neighbourhoods based on the Walk Score metric, taking into account pedestrians' characteristics and amenity attributes (scale and category). 6 categories of amenities were studied, including education, medical care, municipal administration, finance and telecommunication, commercial services, and elderly care. They delivered the questionnaire to 132 respondents to conclude the parameters considered in the metric. They also used a decay factor to account for different age groups etc in terms of walking speed.

\textit{\textbf{Remark:} This paper did not consider any algorithm and modification is heavily based on a questionnaire.}

\section{Other works}

\subsection{A Grammar-Based Optimisation Approach for Designing Urban Fabrics and Locating Amenities for 15-Minute Cities}

This paper uses a geometric grammar based approach to explore computation to support decision-making concerning the layout of urban fabrics and the location of amenities in a neighbourhood \cite{lima_grammar-based_2022}. The authors used an inductive method for qualitative content analysis. However, the authors noted that this solution “does not address irregular or non-orthogonal urban block patterns, and the influence of nearby amenities located outside the studied fabric was not considered.”

\subsection{The Quest for Proximity: A Systematic Review of Computational Approaches towards 15-Minute Cities}

The authors developed a comprehensive overview of the use of computational tools to support the analysis and design of 15-Minute Cities using a systematic literature review \cite{lima_quest_2023}. They noted that the topic of 15-minute city has growth exponentially in popularity and especially in the field of urban design. They concluded that computational approaches to 15-minute city design presents significant challenges such as “Data availability and quality”, “Computational cost” and “Adaptability”.

\subsection{The 15-minute city: Urban planning and design efforts toward creating sustainable neighbourhoods}

The authors collected 103 documents, dealing with underlying principles, sustainability advantages, and critics of the 15-minute city concept \cite{khavarian-garmsir_15-minute_2023}. They defined 7 dimensions constitute the 15-minute city: (1) proximity, (2) density, (3) diversity, (4) digitalisation, (5) human scale
urban design, (6) flexibility, (7) connectivity. The authors summarised the sustainability contributions in social, economical and environmental aspects in the society by 15-Minute City and also the barrier of implementations.

\subsection{The Theoretical, Practical, and Technological Foundations of the 15-Minute City Model: Proximity and Its Environmental, Social and Economic Benefits for Sustainability}

15-Minute City has four main cornerstones (proximity, diversity, density, and digitisation). The authors in this paper explored the proximity dimensions of the 15-Minute City and how it could influence mixed land use to yield environmental, social, and economic benefits \cite{allam_theoretical_2022}.

\subsection{Urban Transition and the Return of Neighbourhood Planning. Questioning the Proximity Syndrome and the 15-Minute City}

The authors of this paper developed an evidence-based approach to a deeper analysis of policy design and implementation of the 15 minute city \cite{marchigiani_urban_2022}. They concluded that the implementation and effectiveness of the 15 minute city depend on the concrete and contextual conformation of each city. The authors ended the paper stating that city development “needs some design framework and structure capable of addressing transformations, and their space and time location and sequences” on top of the “15 minute device” and that “it needs a syntax for an urban planning course of action that is incremental and adaptive but not limited to the contingent, blurred, and agnostic appeal of a catchy label.”