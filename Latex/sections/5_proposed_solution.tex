% Explain how dijkstra fits to fix our problem
% Purpose modification
% Purpose entire solution

Our first proposed solution is to represent the area of interest as a graph. Barbieri et al. proposed to use a planar graph \cite{barbieri_graph_2023}. However, it is not clear if a planar graph would be able to represent bridges/tunnels etc effectively, as the definition of a planar graph is that it can be drawn on the plane in such a way that its edges intersect only at their endpoints.

To formally define the formation of the graph $G(V,E)$ to represent the area of interest to compute 15-Minute City, where $V$ is the set of vertices and $E$ is the set of weighted, undirected edges such that $E\subseteq V\times V$. The vertices of the graph can represent the road junctions and the weight of the edges are proportional to the length of the roads. Location of services can be placed on this graph by adding an extra node on corresponding edge according to the nearest position of the road from the services.

Adapting the notations used by Barbieri et al. \cite{barbieri_graph_2023}, define $f^i$ as the list of vertices in the graph representation the locations of each service type $i=1,...,n$, for $n$ types of services. For each service type, the locations of each service can be denoted as $f^i_j,j=1...,m$, where $m$ denotes the number of services of service type $i$ in the graph.

Subsequently, define $C^i=\bigcup\limits_{j=1}^m C^i_j$ the nodes that can be reached by $f^i$ in less than 15 minutes. More specifically, $C^i_j$ is the set of nodes which can be reached by $f^i_j$ in less than 15 minutes.

$$C^i=\bigcup\limits_{j=1}^m C^i_j\subseteq V$$

If services $f^i_k$, $f^i_l$ of the same type $i$ are far enough, it is possible that $C_k\cap C_l=\emptyset$. This represents a gap in the graph where its nodes cannot reach to service type $i$ within 15 minutes.

Finally, we can define the 15-Minute City as the nodes which can reach to all service types within 15 minutes, namely

$$C=\bigcap\limits_{i=1}^n\bigcup\limits_{j=1}^m C^i_j=\bigcap\limits_{i=1}^n C^i\subseteq V$$

\subsection{Rewrite notations}

Given $n$ types of services, define $F_v\in\{0,1\}^n$ for node $v$ as

$$ F_v(i) = \begin{cases}1&\text{if node }v\text{ can reach to service }i\text{ in 15 minutes}\\0&\text{otherwise}\end{cases}$$

The output of the algorithm is a set $X\subseteq V$ of nodes where

$$\forall v\in X,\enspace F_v=\mathbf{1} $$

\subsection{Search for nodes within 15 minutes}

As we have discussed in \ref{dijkstra}, Dijkstra's algorithm can be used to find the shortest route between 2 locations. Therefore, this can be considered as a motivation/inspiration as part of the solution in finding 15-Minute City.

% Should we include safety factor into weights?
% Directed graph for slopes, so 2 edges between 2 nodes with different weights indicting uphill/downhill.
% Need a formal method to put businesses onto the street

\subsection{Modified Algorithm}

\subsection{Complexity}