\begin{algorithm}[!h]
    \caption{15-Minute City Algorithm}\label{alg:15mc}
    \textbf{Input:} A graph $G(V,E)$, weights $w:E\rightarrow\mathbb{R}_{\geq 0}$, a list $S$ of service vertices of $n$ types and a time threshold $t$\\
    \textbf{Output} Set $R$ of vertices reachable by at most $t$ minutes
    \begin{algorithmic}
        \State $R\gets\emptyset$
        \ForAll{vertex $v \in V$}
            \State $v.r \gets \{\mathbf{0}\}^{n}$
        \EndFor
        \ForAll{service $v \in S$}
            \State $v.l \gets i^{th}$ type of service
        \EndFor
        \For {each service type $i\in\{1,...,n\}$}
            \For {each vertex $s$ where $s.l=i$}
                \State ModifiedDijkstra($G,w,s,t$)
            \EndFor
        \EndFor
        \For {each vertex $v\in V$}
            \If {$\forall i\in\{1,...,n\},\enspace v.r(i)=1$}
                \State $R \gets R\cup \{v\}$
            \EndIf
        \EndFor
    \end{algorithmic}
\end{algorithm}

\begin{algorithm}[!h]
    \caption{ModifiedDijkstra Algorithm}\label{alg:modified_dijsktra}
    \begin{algorithmic}
        \For {each vertex $v\in V$}
            \State $v.d\gets\infty$
        \EndFor
        \State $s.d\gets 0$
        \State $S\gets\emptyset$
        \State $Q\gets\emptyset$
        \For {each vertex $v\in V$}
            \State INSERT($Q,v$)
        \EndFor
        \While {$Q\neq\emptyset$}
            \State $v\gets$EXTRACT-MIN$(Q)$
            \If {$v.d>t$}
                \State \textbf{break}
            \EndIf
            \State $S\gets S\cup\{v\}$
            \For {each vertex $u\in Adj[v]$}
                \If {$u.d>v.d+w(u,v)$}
                    \State $u.d\gets v.d+w(u,v)$
                    \State DECREASE-KEY($Q,u,u.d$)
                \EndIf
            \EndFor
        \EndWhile
    \end{algorithmic}
\end{algorithm}


Similar approach was used by Barbieri et al. \cite{barbieri_graph_2023}, for $n$ types of services. For each service type, the locations of each service can be denoted as $f^i_j,j=1...,m$, where $m$ denotes the number of services of service type $i$ in the graph.

Subsequently, define $C^i=\bigcup\limits_{j=1}^m C^i_j$ the nodes that can be reached by $f^i$ in less than 15 minutes. More specifically, $C^i_j$ is the set of nodes which can be reached by $f^i_j$ in less than 15 minutes.

$$ C^i=\bigcup\limits_{j=1}^m C^i_j\subseteq V$$

If services $f^i_k$, $f^i_l$ of the same type $i$ are far enough, it is possible that $C_k\cap C_l=\emptyset$. This represents a gap in the graph where its nodes cannot reach to service type $i$ within 15 minutes.

Finally, we can define the 15-Minute City as the nodes which can reach to all service types within 15 minutes, namely

$$ C=\bigcap\limits_{i=1}^n\bigcup\limits_{j=1}^m C^i_j=\bigcap\limits_{i=1}^n C^i\subseteq V $$

\subsection{Rewrite notations}

Given $n$ types of services, define $F_v\in\{0,1\}^n$ for node $v$ as

$$ F_v(i) = \begin{cases}1&\text{if node }v\text{ can reach to service }i\text{ in 15 minutes}\\0&\text{otherwise}\end{cases} $$

The output of the algorithm is a set $X\subseteq V$ of nodes where

$$ \forall v\in X,\enspace F_v=\mathbf{1} $$

\subsection{Search for nodes within 15 minutes}

As we have discussed in \ref{dijkstra}, Dijkstra's algorithm can be used to find the shortest route between 2 locations. Therefore, this can be considered as a motivation/inspiration as part of the solution in finding 15-Minute City.

\subsection{Modified Algorithm}

\subsection{Complexity}