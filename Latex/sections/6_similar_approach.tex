Similar approach was used by Barbieri et al. \cite{barbieri_graph_2023}, for $n$ types of services. For each service type, the locations of each service can be denoted as $f^i_j,j=1...,m$, where $m$ denotes the number of services of service type $i$ in the graph.

Subsequently, define $C^i=\bigcup\limits_{j=1}^m C^i_j$ the nodes that can be reached by $f^i$ in less than 15 minutes. More specifically, $C^i_j$ is the set of nodes which can be reached by $f^i_j$ in less than 15 minutes.

$$C^i=\bigcup\limits_{j=1}^m C^i_j\subseteq V$$

If services $f^i_k$, $f^i_l$ of the same type $i$ are far enough, it is possible that $C_k\cap C_l=\emptyset$. This represents a gap in the graph where its nodes cannot reach to service type $i$ within 15 minutes.

Finally, we can define the 15-Minute City as the nodes which can reach to all service types within 15 minutes, namely

$$C=\bigcap\limits_{i=1}^n\bigcup\limits_{j=1}^m C^i_j=\bigcap\limits_{i=1}^n C^i\subseteq V$$

\textbf{Notes:} Barbieri et al. proposed to use a planar graph \cite{barbieri_graph_2023}. However, it is not clear if a planar graph would be able to represent bridges/tunnels etc effectively, as the definition of a planar graph is that it can be drawn on the plane in such a way that its edges intersect only at their endpoints.