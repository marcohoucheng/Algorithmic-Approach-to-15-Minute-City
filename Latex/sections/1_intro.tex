The 15-Minute City was first proposed by Moreno in 2016 as a solution to build safer, more resilient, sustainable and inclusive cities and to harmonise the notion of Smart Cities \cite{moreno_introducing_2021}. Moreno et al. discussed the relationship between the concept of the “15-Minute City” to Urban Planning Pandemic Response, its Emerging Variations (i.e. 20 minute city \cite{capasso_da_silva_accessibility_2019}), walkable neighbour \cite{weng_15-minute_2019} smart cities. 

Moreno et al. proposed the formal “15-Minute City Concept”, arguing that residents should be able to enjoy a higher quality of life where they will be able to effectively fulfil six essential urban social functions to sustain a decent urban life. These include (a) living, (b) working, (c) commerce, (d) healthcare, (e) education and (f) entertainment. The authors then proposed the “modified 15-Minute City” framework, depicting the four identified dimensions that could be incorporated with the already existing one proposed. These are (a) Density, (b) Proximity, (c) Diversity and (d) Digitisation.

Since COVID-19 pandemic, this topic has gained a tremendous amount of attention and has growth exponentially in popularity \cite{lima_quest_2023,allam_theoretical_2022} among various research fields in literature, with urban design being the most studied \cite{lima_quest_2023}. The concept of the 15-Minute City has been studied and shown that it brings benefits to the society (environmental, social, and economical etc.) \cite{allam_theoretical_2022}.

However, many of the studies in 15-Minute City (or $t$ minute city in general) in the literature have used a data-driven approach. Although the 15 minutes threshold is attracted the most attention, the notion of the $t$-Minute City has also been considered \cite{moreno_introducing_2021}. Some examples are the 20-Minute City in Tempe, Arizona \cite{capasso_da_silva_accessibility_2019}, the 30-Minute City in Sydney, Australia \cite{sarkar_measuring_2020} and Olivari et al. studied 5 to 60 Minute City in Ferrara, Italy \cite{olivari_are_2023}. The lack of algorithmic approaches available regarding $t$ minute city means that many solutions available cannot be generalised and to be reapplied to another location of interest. Moreover, Lima et al. has noted that computational approaches to 15-Minute City design presents significant challenges such as “data availability and quality”, “computational cost” and “adaptability” \cite{lima_quest_2023}. Notably, Weng et al. \cite{weng_15-minute_2019} and Olivari et al.\cite{olivari_are_2023} relied on census data on population density for their respective studies which may not be available for some countries and regions. 