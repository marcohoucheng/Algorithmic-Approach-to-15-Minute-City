In the previous section \ref{previous_works}, serveral studies have discussed the needs for standardisation in terms of the methodogy used in computing the 15-Minute City. In particular, Pezzica et al. argued that \say{the absence of standardised modelling protocols imposes significant limitations on the application of MC models, hinders synchronic comparisons, and can indirectly foster the formulation of exclusive policies and inconsistent planning decisions} \cite{Pezzica_Altafini_Mara_Chioni_2024}. Lima et al. noted that \say{introducing computational approaches to 15-minute city design presents significant challenges and potential bottlenecks. On the other hand, exploring these challenges as opportunities for inserting new research is also possible since the theme is rising}, some of these challenges are “Data availability and quality”, “Computational cost” and “Adaptability” \cite{lima_quest_2023}. Furthermore, Marchigiani et al. found that the approach to 15-Minute City needs to be changed and adapted to each location case by case \cite{marchigiani_urban_2022}.

With these in mind, the aim of this thesis...

